\begin{abstract}
%Abstract must be capable of substituting for the whole thesis when there is insufficient time and space for the full text.
%The structure of the abstract should mirror the structure of the whole thesis, and should represent all its major elements.


The interaction between two adjacent cylinders immersed in fluid is studied numerically by solving the two-dimensional Navier-Stokes (NS) equations using a finite element method (FEM). The two rigid cylinders are immersed in otherwise stationary fluid. One cylinder (C1) undergoes forced vibration that disturbs the fluid, whereas another cylinder (C2), which has one-degree-of-freedom (1DOF), vibrates correspondingly under the action of the imbalanced hydrodynamic force. All the simulations carried out in this study have a constant Reynolds number of 100 and a constant mass ratio of 2.5. Simulations are conducted with a series gap ratios, G, ranging from 0.025 to 3 times of the cylinder diameter, with various forced vibration amplitude $ A_1 $ and frequency $ f_1 $ of C1. We find that C2's amplitude ($ A_2 $) reaches maximum when C1's vibration frequency ($ f_1 $) is approximately 80\% of C2's structural natural frequency ($ f_n $). The increase of C1's amplitude ($ A_1 $) leads to the rise of C2's amplitude ($ A_2 $) but the decrease of its relative amplitude ($ A_2 /A_1$). The increase in $ A_1 $ also slightly reduces the value of $ f_1 $ at the maximum $ A_2 $. In addition, the vibration centre of C1 shifts away from the initial location in the cases with large gap ratio. The shift becomes increasingly obvious with an increase of $ A_1 $ and $ f_1 $. Finally, C2 responds to the vibration of C1 with bigger amplitude in some situations with small gap ratios, which can be interpreted as the occurrence of resonance. 
\\
\\
\textbf{Key words:} Vortex-induced vibration, computational fluid dynamics, resonance 


\end{abstract}